El objetivo de este trabajo práctico es el desarrollo y evaluación de una herramienta de OCR (Optical Character Recognition) para el reconocimiento de dígitos manuscritos en imágenes.

Para realizar el reconocimiento se utilizará como entrenamiento del programa una base de datos de imágenes ya etiquetadas con el dígito correcto. Las imágenes a reconocer serán comparadas con las de esta base de datos y se les asignará una clase ---un dígito--- usando el algoritmo de k-NN (k Nearest Neighbours).

Dado que el gran tamaño de la base de entrenamiento hace al algoritmo de k-NN muy ineficiente, se hará uso de la técnica de PCA (Principal Component Analysis) para reducir el costo temporal del programa.

\section{k-NN}
El algoritmo de k-NN consiste en obtener las $k$ imágenes de la base de entrenamiento más cercanas a la imagen a reconocer ---los $k$ vecinos más cercanos--- para luego asignarle una clase según algún criterio. En esta implementación se decidió elegir la clase a la cual pertenece la mayor cantidad de estas imágenes. En caso de haber más de una, se elige a aquella que tenga una imagen más cercana a la imagen a reconocer.


\section{PCA}
PCA es un método en el cual se proyectan las imágenes vectorizadas a una determinada base de tal manera que queden las componentes que aportan más información al comienzo del vector. Luego, en lugar de trabajar con todas las componentes, se puede considerar únicamente una cierta cantidad $\alpha$ de las primeras coordenadas descartando el resto ya que no poseen tanta información.

La base mencionada será la base de autovectores de la matriz de covarianza de los datos de entrenamiento, ordenados de forma decreciente con respecto a la magnitud de los autovalores asociados. Luego la "importancia" de las componentes se corresponde con la magnitud del autovalor correspondiente a esa coordenada.