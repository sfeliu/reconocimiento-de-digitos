El desarrollado de este trabajo pretende hacer un algoritmo de clasificación de aprendizaje supervisado, esté se puede dividir en dos partes: el entrenamiento de la base de datos y el reconocimiento del dígito de una imagen. \par 
\indent Con este objetivo, lo primero fue implementar el algoritmo knn para poder hacer una clasificación de los datos y tener una primera impresión de como se comportaria el algoritmo kNN. \par
\indent Dado el problema de la dimensionalidad del algoritmo kNN , se considero hacer una reducción de las dimensóon de los datos para hacer la clasificación de los datos con un menor costo, siendo una algoritmo robusto y con la menor perdidad de información durante la reducción. Dadas estas condiciones, usaremos PCA para poder hacer la reducción de las dimensiones. \par


\section{Entrenamiento}

\subsection{Algoritmo Knn}
Comenzamos buscando la norma de la distancia euclídea entre la imagen a reconocer y cada uno de los vectores de la base de entrenamiento. Para cada norma $d_{i}=||X-Y_{i}||$, donde X es la imagen a reconocer e $Y_{i}$ es el $i$-esimo vector de la base de entrenamiento, creamos un tupla que tiene la norma junto con el digito manuscrito asociado a $Y_{i}$.\par
\indent Los primeros k tuplas son insertadas en un max heap independientemente del tamaño de la norma que tengan. Para el tupla k+1 vemos si el tamaño de su norma es menor al tamaño de la norma en el tupla de la cabeza del max heap. Si esto sucede borramos el elemento en la cabeza del heap e insertamos el tupla con la norma $d_{k+1}$, de lo contrario el heap no se cambia e iteramos sobre $i$ hasta haber utilizado todas las imágenes del dataset de entrenamiento. \par
\indent Con el max heap de las $k$ normas mas pequeñas construido el siguiente paso es encontrar el dígito manuscrito mas repetido entre todos los tupla del heap. De aquí se obtendrá la clase que tenga una mayor cantidad de vectores dentro de los k más cercanos.\par
\indent En el caso de encontrar clases con la misma cantidad de vectores se compararán los elementos de cada una de ellas en orden de cercanía, para obtener así la clase que tenga al elemento más cercano.\par
%Dada una metrica definida para los datos, el algoritmo toma un conjunto
%elementos conformado por los k vecinos mas cercanos segun la metrica y toma la clase moda del conjunto como la clase del elemento. 

\subsection{Normalización de los datos}
Los datos seran centrados en 0, para poder tener una mejor representación de los
datos y poder trabajar con los datos concervando la varianza, para luego poder analizar
los datos con PCA

\subsection{Obtención de la Matriz de Covarianza}
Primero se vectorizan los datos definiendo una matriz X, entonces la matriz de covarianza queda definida como $M_{x}$ = $\frac{1}{n-1}$ $X^{t}$ X, con lo cual M es simétrica y se puede formar una BON(Base Ortonormal) de autovectores. Esto sirven para que el algoritmo PCA pueda trabajar con los datos conservando la misma varianza,y asi hallar los autovectores que al proyectar un dato sobre ellos concerven la mayor varianza.

\subsection{PCA}
Para el procedimiento PCA se utilizaran el método de las potencias y deflación. 
\begin{enumerate}
\item El método de la potencia lo utilizamos para obtener el autovector con el mayor autovalor asociado ---comparandolo al resto de los autovalores en módulo. La diferencia al método visto en el $libro^{1}$, es que como no tenemos tiempo infinito para correr tomamos el criterio de deje de iterar al momento que el autovalor obtenido en $t_{i}$ tenga una diferencia menor a un $\epsilon$ con respecto al autovalor obtenido en $t_{i-1}$, con $\epsilon$ muy pequeño.
\item Dado que la matriz de covarianza es simétrica, entonces los autovectores son ortogonales de a pares. Al aplicar deflación, se toma el autovector asociado al mayor autovalor y se le asocia el 0 como autovalor, preservando los demas autovectores con sus respectivos autovalores, con lo cual, se puede aplicar nuevamente el método de las potencias hasta obtener la cantidad de autovectores deseado para la proyección al espacio que maximiza la varianza.

\end{enumerate}
%Para el algoritmo PCA se utilizaran dos algoritmos: 
%1ro: Metodos de las potencias, el cual usa convergencia de las potencias de
%la matriz por un vector random al autovector y  autovalor(mayor en modulo a los
%demas autovalores)
%2do: Deflacion, usa la propoedad de que cuando se armar una matriz con el autovalor
%y la multiplicacion del autovector por su traspuesto y restamos esa matriz a la
%original, se esta cambiando el valor del autovalor asociado del autovector que
%ahora pasa a ser 0.
%Para tener una mejor representacion de los datos usaremos la matriz de
%covarianza de los datos y el algoritmo PCA dara sus componentes principales. 


%\subsection{Deflaccion}
%Dado que la matriz de covarianza es simetrica, entonces los autovectores son ortogonales de a pares. Al aplicar deflacion, se toma el autovector asociado al mayor autovalor y se le asocia el 0 como autovalor, preservando los demas autovectores con sus respectivos autovalores, con lo cual, se puede aplicar nuevamente el metodo de las potencias hasta obtener la cantidad de autovectores deseado para la proyeccion al espacio que maximiza la varianza.

\subsection{Utilización de k-NN con PCA}

Antes de poder utilizar la herramienta de OCR que implementamos tenemos que decidir el tamaño de los parametros $\alpha$ y $k$ que vamos a utilizar. El parametro $k$ nos dice el tamaño de la vecindad que concidera el algoritmo kNN y $\alpha$ nos dice cuantos valores principales utilizar cuando aplicamos PCA. \par
\indent El procedimiento PCA primero calcula la matriz de covarianza de los datos de entrenamiento y despues calcula y forma una matriz $V$ con los primeros $\alpha$ autovectores ---asociados a los componentes principales--- de esta matriz utilizando el método de la potencia y el algoritmo de deflación. \par
\indent Por último se vectorizan las imágenes de los datasets test y entrenamiento para despues aplicar el cambio de base $tc(X_{1})=X_{1}*V$ y $tc(X_{2})=X_{2}*V$, donde $X_{1}$ y $X_{2}$ son las matrices que tiene como filas las imágenes de los datasets train y test respectivamente. \par
\indent Una vez finalizado el proceso PCA se aplica el algoritmo kNN sobre  $tc(X_{1})$ y $tc(X_{2})$. La idea de hacer todo el procedimiento de PCA es encontrar un $\alpha$ óptimo tal que nos deje minimizar el tamaño de nuestros datasets con fines de mejorar el tiempo de cómputo sin perder información.\par
