El programa desarrollado se puede dividir en dos partes principales: el entrenamiento de la base de datos y el reconocimiento de los dígitos.

\section{Entrenamiento}
En esta sección se intentan transformar los datos para obtener una reducción de tiempo de ejecución al realizar el reconocimiento de los digitos. 

Para ello se aplica el método de PCA a las imágenes de la base de entrenamiento. La elección de componentes se realiza utilizando como criterio la matriz de covarianza de las imágenes, obteniendo así un cambio de base óptimo para anlizar las mayores diferencias entre las imágenes.

\subsection{Matriz de covarianza}

Lo que se quiere lograr al obtener la matriz de covarianza es obtener la distancia de la media de cada píxel en particular.

Para conseguir la matriz se vectorizan todas las imágenes de entrenamiento y se las coloca por fila en una matriz. De esta forma, las filas corresponderán a las imágenes de la base y las columnas a los píxeles de cada imagen.

Luego se obtiene la media de cada columna y se le resta este valor a cada fila. La matriz resultante la notaremos como $X$ y nuestra matriz de covarianza será $M_{X}$ la cual se definirá como $M_{X} = \frac{1}{n-1} X^{t} X$

\subsection{Autovalores y autovectores}

Una vez obtenida la matriz de covarianza queremos conseguir los autovalores más significativos, porque estos tendrían asociados los autovectores que se utilizarán para hacer el cambio de base. 

Para encontrar estos autovalores se utiliza el método de la potencia, que al tender la cantidad de iteraciones a infinito, logra emerger el autovalor dominante y el autovector asociado de la matriz al que se lo aplica.

Como esto sólo obtiene el autovalor dominante y no el resto, debemos utilizar el método de deflación para "quitarle" el autovalor mencionado y poder obtener los demás.

\subsection{Matriz de cambio de base}

Habiendo decidido cuántos autovalores obtendremos del método de la potencia, colocaremos los autovectores asociados a estos autovalores en filas, formando una matriz de cambio de base, donde las filas mantendrán el orden en el cual fueron obtenidos los autovectores, es decir, ordenado decrecientemente según la magnitud de los autovalores asociados.

Por último, se aplica este cambio de base a la base de entrenamiento.


\section{Reconocimiento de dígitos}

\subsection{Cambio de base}

Para poder utilizar k-NN debemos aplicarle a las imágenes de a reconocer las mismas transformaciones que se le realizaron a la base de entrenamiento. Primero se las normalizará restándoles la media obtenida durante el entrenamiento y se las dividirá por $\sqrt{n - 1}$, siendo $n$ la cantidad de filas en $X$. Finalmente se le aplica la matriz de cambio de baso.

\subsection{k-NN}

Para obtener la clase de las imágenes utilizaremos k-NN.

Comenzaremos buscando la norma de la distancia euclídea entre la vectorización de la imagen a reconocer y cada uno de los vectores de la base de entrenamiento. Aquí se puede observar la razón de la optimización de recortar la dimensión de los vectores con PCA.

Una vez obtenida todas las distancias se tomarán los k vectores más cercanos. De aquí se obtendrá la clase que tenga una mayor cantidad de vectores dentro de estos k más cercanos. En el caso de encontrar clases con la misma cantidad de vectores se compararán los elementos de cada una de ellas en orden de cercanía, para obtener así la clase que tenga al elemento más cercano.