Para la experimentación debemos utilizar solo una parte de nuestra base de entrenamiento, así poder saber cuales son nuestros errores y donde suelen encontrarse.

Por lo tanto se utilizará la técnica de K-Fold cross validation, que basicamente es dividir el conjunto total de nuestros datos en K subconjuntos, de los cuales se usaran K-1 subconjuntos para entrenamiento y el subconjunto restante sera el de validación. Esto se devera hacer K veces cambiando el subconjunto de validación por uno de entrenamiento, sin repetir la utilización de ellos.

De esta manera se obtendran los valores promedios de efectividad para cada combinación distinta de parametros.