Luego de realizar todos los experimentos y haber entendido bien el funcionamiento de las partes individuales del OCR, podemos realizar algunas conclusiones.

\begin{enumerate}
	\item Con el primer experimento pudimos observar los valores óptimos para las variables, los cuales fueron .... y nuestras hipótesis se cumplieron. 
	\begin{itemize}
		\item Estas hipótesis cubrian desde que con k-KNN muy grandes se tomaban en cuenta tantas imágenes distintas que se perdian los buenos resultados, hasta que con $\alpha$ grandes no se encontraba una diferencia abrupta entre las métricas obtenidas.
		\item También encontramos que al tomar $\alpha$ cada vez mas grande, el tiempo en ejecución crecia linealmente. Entonces si se toma nuestro $\alpha$ óptimo nos da buenos resultados y a su vez el tiempo de ejecución es considerablemente bueno.
	\end{itemize}
	\item Al comparar los resultados obtenidos con dos métricas distintas encontramos en particular que las relaciones entre estas se mantuvieron proporcionalmente iguales. No obstante, los porcentajes obtenidos, al tomar distintos parametros, fueron mas dispersos con F1 score que con accuracy. El accuracy esta definido como $\frac{tp + tn}{tp + fp + tn + fn}$ en donde al usar esta métrica en un clasificador multiclase el $tn$ generalmente tenderá a ser el mayor valor, y si no lo es para una clase, lo seran para las otras clase, entonces al hacer un promedio el resultado será un porcentaje alto. En cambio, para el F1 score, se encuentra un comportamiento distinto, este se define como 2$\frac{precisionXrecall}{precision + recall}$ con el cual se obtendra el compromiso entre el recall y el precision, entonces con el cambio de valor de alguno, se encontrará una diferencia mayor.
	\item Al momento de elegir un k-KNN es importante relacionarlo con el tamaño de la base de entrenamiento que poseamos ---por mas que mientras mayor $k$ peor resultado. Esto se debe a que con bases de entrenamiento pequeños se tomaran mayores proporciones de esta base a medida que agrandamos el $k$, y los resultados nos decreceran mas rápido. De esta manera sabemos que si tomamos el $k$ igual a la cantidad de imágenes de entrenamiento, las predicciones siempre seran a la clase con mas imágenes, y en el caso de que haya la misma cantidad, la predicción sera la misma que al hacer $k = 1$


\end{enumerate}