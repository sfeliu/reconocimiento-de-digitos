Luego de realizar todos los experimentos y haber entendido bien el funcionamiento de las partes individuales del OCR, podemos realizar algunas conclusiones.

\begin{enumerate}
	\item Con el primer experimento pudimos observar los valores óptimos para las variables, los cuales fueron .... y nuestras hipótesis se cumplieron. Estas hipótesis cubrian desde que con $\alpha$ grandes no se encontraba una diferencia abrupta entre los resultados, hasta que con kNN muy grandes se tomaban en cuenta tantas imágenes distintas que se perdian los buenos resultados.
	\item Al comparar los resultados obtenidos con dos métricas distintas encontramos en particular que las relaciones entre estas se mantuvieron proporcionalmente. No obstante, los porcentajes obtenidos, al tomar distintos parametros, fueron mas dispersos con F1 score que con accuracy. El accuracy esta definido como $\frac{tp + tn}{tp + fp + tn + fn}$ en donde al usar esta métrica en un clasificador multiclase el tn generalmente tenderá a ser el mayor valor, y si no lo es para una clase, lo seran para las otras clase, entonces al hacer un promedio el resultado será un porcentaje alto. En cambio, para el F1 score, se encuentra un comportamiento distinto, este se define como 2$\frac{precisionXrecall}{precision + recall}$ con el cual se obtendra el compromiso entre el recall y el precision, entonces con el cambio de valor de alguno, se encontrará una diferencia mayor.
	\item 

\end{enumerate}