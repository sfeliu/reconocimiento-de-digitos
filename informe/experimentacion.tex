En esta seccion disenamos experimentos para encontrar parametros optimos  de kNN y PCA de la herramienta de OCR implementado. Para poder hacer esto utilizamos K-fold cross validation.
Como el tiempo de computo de los algoritmos son altos primero hicimos una exploracion manualmente con algunos parametros dispersos con el fin de acotar el espacio de busqueda. Las metricas que utilizamos para analizar que tan bien se reconocen los caracteres.

Para el algoritmo de KNN, nuestra hipotesis es que para los valores mas bajos de k la cantidad de aciertos sera menor porque para caracteres similares una imagen puede tener algunos vecinos cercanos de otra clase. Para valores muy grandes de k se consideran demasiados digitos y en este caso toma mas importancia a la cantidad de apariciones de un caracter que la cercania de los mismbros de su clase. Un ejemplo para ilustrar este comportamiento es cuando k toma su valor maximo, la cantidad total de datos de entrenamiento, la repuesta siempre sera la clase que tenga mas elementos en la base de entrenamiento.

Para los algotimos del metodo PCA, nuestra hipotesis es que la caldidad de nuestos resultados crece junto con el valor de alpha, aunque creemos que va a existir un punto b en el cual los resultados no van tener mejoras significativas. Creemos que esto se debe a que proyectamos el espacio original a otro generado por los autovectores asociados a los mayores autovalores asi se preserva la mayor cantidad de informacion en las primeras b columnas de la matriz transformada.